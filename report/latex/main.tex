\documentclass{article}
\usepackage{homework}
\usepackage{graphicx}

\begin{document}

\hwtitle{ {\tt psychic\_longbow}: another solver for the Traveling Salesman Problem}{CS 5600}{Parker Michaelson}

\namesection{Introduction}

To quote the {\tt psychic\_archer} report:

\begin{quote}
The Traveling Salesman Problem (TSP) is a classic problem in computer science with many immediate theoretical and practical applications.
In its original statement, it takes the form of a traveling salesman attempting to find the shortest route through a group of cities which visits each city exactly once {\it and} returns to the city in which the salesman started.
More formally the solution to the TSP is the cheapest Hamiltonian cycle through a complete graph, with each node having a defined ``cost'' in-between itself and all other nodes. \\
\end{quote}

Using classical (non-AI) methods, the TSP is an NP-hard problem, with a solution time of \BigO{n!}.
By comparison, the AI techniques used in {\tt psychic\_longbow} generally do not find the optimum solution.
Instead, the hill-climbing AI algorithms used in {\tt psychic\_longbow} are given a time budget in which the algorithm is guaranteed to terminate.
Many possible solutions are then iteratively examined, each one derived from the last using an operator.
At the end of the time budget (or possibly sooner depending on the specific algorithm), the algorithm returns the best solution yet found.
The primary difference between algorithms is the manner in which each solution is derived from the previous.

{\tt psychic\_longbow} contains the implementation of the hill-climbing and simulated-annealing algorithms, with three operators.

\end{document}
